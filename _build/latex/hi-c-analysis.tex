%% Generated by Sphinx.
\def\sphinxdocclass{report}
\documentclass[letterpaper,10pt,english]{sphinxmanual}
\ifdefined\pdfpxdimen
   \let\sphinxpxdimen\pdfpxdimen\else\newdimen\sphinxpxdimen
\fi \sphinxpxdimen=.75bp\relax
\ifdefined\pdfimageresolution
    \pdfimageresolution= \numexpr \dimexpr1in\relax/\sphinxpxdimen\relax
\fi
%% let collapsible pdf bookmarks panel have high depth per default
\PassOptionsToPackage{bookmarksdepth=5}{hyperref}

\PassOptionsToPackage{booktabs}{sphinx}
\PassOptionsToPackage{colorrows}{sphinx}

\PassOptionsToPackage{warn}{textcomp}
\usepackage[utf8]{inputenc}
\ifdefined\DeclareUnicodeCharacter
% support both utf8 and utf8x syntaxes
  \ifdefined\DeclareUnicodeCharacterAsOptional
    \def\sphinxDUC#1{\DeclareUnicodeCharacter{"#1}}
  \else
    \let\sphinxDUC\DeclareUnicodeCharacter
  \fi
  \sphinxDUC{00A0}{\nobreakspace}
  \sphinxDUC{2500}{\sphinxunichar{2500}}
  \sphinxDUC{2502}{\sphinxunichar{2502}}
  \sphinxDUC{2514}{\sphinxunichar{2514}}
  \sphinxDUC{251C}{\sphinxunichar{251C}}
  \sphinxDUC{2572}{\textbackslash}
\fi
\usepackage{cmap}
\usepackage[T1]{fontenc}
\usepackage{amsmath,amssymb,amstext}
\usepackage{babel}



\usepackage{tgtermes}
\usepackage{tgheros}
\renewcommand{\ttdefault}{txtt}



\usepackage[Bjarne]{fncychap}
\usepackage{sphinx}

\fvset{fontsize=auto}
\usepackage{geometry}


% Include hyperref last.
\usepackage{hyperref}
% Fix anchor placement for figures with captions.
\usepackage{hypcap}% it must be loaded after hyperref.
% Set up styles of URL: it should be placed after hyperref.
\urlstyle{same}

\addto\captionsenglish{\renewcommand{\contentsname}{Contents:}}

\usepackage{sphinxmessages}
\setcounter{tocdepth}{1}



\title{Hi\sphinxhyphen{}C\sphinxhyphen{}Analysis}
\date{Mar 19, 2023}
\release{0.1.0}
\author{Lorenzo Barsotti}
\newcommand{\sphinxlogo}{\vbox{}}
\renewcommand{\releasename}{Release}
\makeindex
\begin{document}

\ifdefined\shorthandoff
  \ifnum\catcode`\=\string=\active\shorthandoff{=}\fi
  \ifnum\catcode`\"=\active\shorthandoff{"}\fi
\fi

\pagestyle{empty}
\sphinxmaketitle
\pagestyle{plain}
\sphinxtableofcontents
\pagestyle{normal}
\phantomsection\label{\detokenize{index::doc}}



\chapter{Preprocessing Module}
\label{\detokenize{index:preprocessing-module}}

\section{The create\_graph function}
\label{\detokenize{index:the-create-graph-function}}\index{create\_graph() (in module hicanalysis.preprocessing)@\spxentry{create\_graph()}\spxextra{in module hicanalysis.preprocessing}}

\begin{fulllineitems}
\phantomsection\label{\detokenize{index:hicanalysis.preprocessing.create_graph}}
\pysigstartsignatures
\pysiglinewithargsret{\sphinxcode{\sphinxupquote{hicanalysis.preprocessing.}}\sphinxbfcode{\sphinxupquote{create\_graph}}}{\emph{\DUrole{n}{matrix\_df}\DUrole{p}{:}\DUrole{w}{  }\DUrole{n}{DataFrame}}, \emph{\DUrole{n}{metadata\_df}\DUrole{p}{:}\DUrole{w}{  }\DUrole{n}{DataFrame}}}{{ $\rightarrow$ Graph}}
\pysigstopsignatures
\sphinxAtStartPar
This function creates a nx.Graph object, starting from the adjacency matrix of the network and the label of each node.


\subsection{Parameters}
\label{\detokenize{index:parameters}}\begin{quote}
\begin{description}
\sphinxlineitem{matrix\_df (pd.DataFrame): }
\sphinxAtStartPar
Data Frame containing the adjacency matrix of the network.

\sphinxlineitem{metadata\_df (pd.DataFrame): }
\sphinxAtStartPar
Data Frame containing the metadata of the nodes.

\end{description}
\end{quote}


\subsection{Returns}
\label{\detokenize{index:returns}}\begin{quote}
\begin{description}
\sphinxlineitem{nx.Graph: }
\sphinxAtStartPar
Graph object of NetworkX library representing described by the adjacency contained in the \sphinxtitleref{matrix\_df}.

\end{description}
\end{quote}

\end{fulllineitems}



\section{The remove\_empty\_axis function}
\label{\detokenize{index:the-remove-empty-axis-function}}\index{remove\_empty\_axis() (in module hicanalysis.preprocessing)@\spxentry{remove\_empty\_axis()}\spxextra{in module hicanalysis.preprocessing}}

\begin{fulllineitems}
\phantomsection\label{\detokenize{index:hicanalysis.preprocessing.remove_empty_axis}}
\pysigstartsignatures
\pysiglinewithargsret{\sphinxcode{\sphinxupquote{hicanalysis.preprocessing.}}\sphinxbfcode{\sphinxupquote{remove\_empty\_axis}}}{\emph{\DUrole{n}{matrix}\DUrole{p}{:}\DUrole{w}{  }\DUrole{n}{ndarray}}}{{ $\rightarrow$ ndarray}}
\pysigstopsignatures
\sphinxAtStartPar
This function returns the given matrix removing the rows and columns that contain only 0.


\subsection{Parameters}
\label{\detokenize{index:id1}}\begin{quote}
\begin{description}
\sphinxlineitem{matrix (np.ndrray): }
\sphinxAtStartPar
original matrix from which remove empty rows and columns (containing all zeros).

\end{description}
\end{quote}


\subsection{Returns}
\label{\detokenize{index:id2}}\begin{quote}
\begin{description}
\sphinxlineitem{np.ndarray: }
\sphinxAtStartPar
submatrix of the original one, where the empty rows and columns have been removed.

\end{description}
\end{quote}

\end{fulllineitems}



\section{The extract\_diagonal\_blocks function}
\label{\detokenize{index:the-extract-diagonal-blocks-function}}\index{extract\_diagonal\_blocks() (in module hicanalysis.preprocessing)@\spxentry{extract\_diagonal\_blocks()}\spxextra{in module hicanalysis.preprocessing}}

\begin{fulllineitems}
\phantomsection\label{\detokenize{index:hicanalysis.preprocessing.extract_diagonal_blocks}}
\pysigstartsignatures
\pysiglinewithargsret{\sphinxcode{\sphinxupquote{hicanalysis.preprocessing.}}\sphinxbfcode{\sphinxupquote{extract\_diagonal\_blocks}}}{\emph{\DUrole{n}{matrix}\DUrole{p}{:}\DUrole{w}{  }\DUrole{n}{ndarray}}, \emph{\DUrole{n}{block\_size}\DUrole{p}{:}\DUrole{w}{  }\DUrole{n}{list}}}{{ $\rightarrow$ list}}
\pysigstopsignatures
\sphinxAtStartPar
This function extract the diagonal blocks of dimension contained in the 
block\_size, and return them under the form of a list of np.ndarray.


\subsection{Parameters}
\label{\detokenize{index:id3}}\begin{quote}
\begin{description}
\sphinxlineitem{matrix (np.ndarray): }
\sphinxAtStartPar
initial matrix from which we want to extract the blocks of the diagonal

\sphinxlineitem{block\_size (list): }
\sphinxAtStartPar
list containing the dimension of each block in the diagonal.

\end{description}
\end{quote}


\subsection{Returns}
\label{\detokenize{index:id4}}\begin{quote}
\begin{description}
\sphinxlineitem{list: }
\sphinxAtStartPar
list of np.ndarray containing the blocks of the diagonal extracted from the original matrix.

\end{description}
\end{quote}

\end{fulllineitems}



\section{The get\_chromosome\_list function}
\label{\detokenize{index:the-get-chromosome-list-function}}\index{get\_chromosome\_list() (in module hicanalysis.preprocessing)@\spxentry{get\_chromosome\_list()}\spxextra{in module hicanalysis.preprocessing}}

\begin{fulllineitems}
\phantomsection\label{\detokenize{index:hicanalysis.preprocessing.get_chromosome_list}}
\pysigstartsignatures
\pysiglinewithargsret{\sphinxcode{\sphinxupquote{hicanalysis.preprocessing.}}\sphinxbfcode{\sphinxupquote{get\_chromosome\_list}}}{\emph{\DUrole{n}{metadata\_df}\DUrole{p}{:}\DUrole{w}{  }\DUrole{n}{DataFrame}}}{{ $\rightarrow$ list}}
\pysigstopsignatures
\sphinxAtStartPar
This function, given a dataframe as input, containing the columns \sphinxtitleref{start} and \sphinxtitleref{end}, containing the 
number of row of start and end of each chromosome, returns a list with the difference between 
\sphinxtitleref{end} element and \sphinxtitleref{start} element.


\subsection{Parameters}
\label{\detokenize{index:id5}}\begin{quote}
\begin{description}
\sphinxlineitem{metadata\_df (pd.DataFrame): }
\sphinxAtStartPar
Dataframe containing a column named \sphinxtitleref{start}, a column \sphinxtitleref{end} and a 
column with the name of the chromosome (not necessary).

\end{description}
\end{quote}


\subsection{Returns}
\label{\detokenize{index:id6}}\begin{quote}
\begin{description}
\sphinxlineitem{list: }
\sphinxAtStartPar
list containing the number of segments into which a chromosome has been divided.

\end{description}
\end{quote}

\end{fulllineitems}



\section{The ooe\_normalization function}
\label{\detokenize{index:the-ooe-normalization-function}}\index{ooe\_normalization() (in module hicanalysis.preprocessing)@\spxentry{ooe\_normalization()}\spxextra{in module hicanalysis.preprocessing}}

\begin{fulllineitems}
\phantomsection\label{\detokenize{index:hicanalysis.preprocessing.ooe_normalization}}
\pysigstartsignatures
\pysiglinewithargsret{\sphinxcode{\sphinxupquote{hicanalysis.preprocessing.}}\sphinxbfcode{\sphinxupquote{ooe\_normalization}}}{\emph{\DUrole{n}{matrix}\DUrole{p}{:}\DUrole{w}{  }\DUrole{n}{ndarray}}}{{ $\rightarrow$ ndarray}}
\pysigstopsignatures
\sphinxAtStartPar
This function allows to normalize a given matrix with the Observed\sphinxhyphen{}Over\sphinxhyphen{}Expected (OOE) 
algorithm for a symmetric matrix. In particular this algorithm provides to divide each 
element for the mean of the diagonal it belongs to.


\subsection{Parameters}
\label{\detokenize{index:id7}}\begin{quote}
\begin{description}
\sphinxlineitem{matrix (np.ndarray): }
\sphinxAtStartPar
symmetric matrix to be normalized.

\end{description}
\end{quote}


\subsection{Returns}
\label{\detokenize{index:id8}}\begin{quote}
\begin{description}
\sphinxlineitem{np.ndarray:     }
\sphinxAtStartPar
matrix normalized with the OOE algorithm.

\end{description}
\end{quote}

\end{fulllineitems}



\section{The build\_projectors function}
\label{\detokenize{index:the-build-projectors-function}}\index{build\_projectors() (in module hicanalysis.preprocessing)@\spxentry{build\_projectors()}\spxextra{in module hicanalysis.preprocessing}}

\begin{fulllineitems}
\phantomsection\label{\detokenize{index:hicanalysis.preprocessing.build_projectors}}
\pysigstartsignatures
\pysiglinewithargsret{\sphinxcode{\sphinxupquote{hicanalysis.preprocessing.}}\sphinxbfcode{\sphinxupquote{build\_projectors}}}{\emph{\DUrole{n}{eigenvectors}\DUrole{p}{:}\DUrole{w}{  }\DUrole{n}{array}}, \emph{\DUrole{n}{number\_of\_projector\_desired}\DUrole{p}{:}\DUrole{w}{  }\DUrole{n}{int}}}{{ $\rightarrow$ list}}
\pysigstopsignatures
\sphinxAtStartPar
This function allows to get a list containing the projectors corresponding to the eigenvectors provided
as input. The eigenvectors must be contained in a np.array where the columns are the eigenvectors.


\subsection{Parameters}
\label{\detokenize{index:id9}}\begin{description}
\sphinxlineitem{eigenvectors (np.array) :}
\sphinxAtStartPar
Array containing the in the columns the eigenvectors of a matrix.

\sphinxlineitem{number\_of\_projector\_desired (int) :}
\sphinxAtStartPar
Number of projectors that you want to build.

\end{description}


\subsection{Returns}
\label{\detokenize{index:id10}}\begin{description}
\sphinxlineitem{list: }
\sphinxAtStartPar
Output list containing the projectors corresponding to the input eigenvectors.

\end{description}

\end{fulllineitems}



\section{The reconstruct\_matrix function}
\label{\detokenize{index:the-reconstruct-matrix-function}}\index{reconstruct\_matrix() (in module hicanalysis.preprocessing)@\spxentry{reconstruct\_matrix()}\spxextra{in module hicanalysis.preprocessing}}

\begin{fulllineitems}
\phantomsection\label{\detokenize{index:hicanalysis.preprocessing.reconstruct_matrix}}
\pysigstartsignatures
\pysiglinewithargsret{\sphinxcode{\sphinxupquote{hicanalysis.preprocessing.}}\sphinxbfcode{\sphinxupquote{reconstruct\_matrix}}}{\emph{\DUrole{n}{projectors}\DUrole{p}{:}\DUrole{w}{  }\DUrole{n}{list}}, \emph{\DUrole{n}{number\_of\_projectors}\DUrole{p}{:}\DUrole{w}{  }\DUrole{n}{int}}}{{ $\rightarrow$ ndarray}}
\pysigstopsignatures
\sphinxAtStartPar
This function takes as input a list of matrices (projectors) and a number of projectors to be added
together in order to reconstruct the matrix.


\subsection{Parameters}
\label{\detokenize{index:id11}}\begin{description}
\sphinxlineitem{projectors (list) :}
\sphinxAtStartPar
List containing np.ndarray of the projectors. It is supposed that the projectors are ordered 
from the most to the least important.

\sphinxlineitem{number\_of\_projectors (int) :}
\sphinxAtStartPar
Number of projectors to be added together.

\end{description}


\subsection{Returns}
\label{\detokenize{index:id12}}\begin{description}
\sphinxlineitem{np.ndarray: }
\sphinxAtStartPar
Reconstructed matrix from the addition of the first \sphinxtitleref{number\_of\_projectors} projectors.

\end{description}


\subsection{Raises}
\label{\detokenize{index:raises}}\begin{quote}
\begin{description}
\sphinxlineitem{ValueError: }
\sphinxAtStartPar
When the \sphinxtitleref{number\_of\_projectors} parameter is greater than the length of the \sphinxtitleref{projectors} list.

\end{description}
\end{quote}

\end{fulllineitems}



\chapter{Visualizegraph Module}
\label{\detokenize{index:visualizegraph-module}}

\section{The weights\_distribuion function}
\label{\detokenize{index:the-weights-distribuion-function}}\index{weights\_distribuion() (in module hicanalysis.visualizegraph)@\spxentry{weights\_distribuion()}\spxextra{in module hicanalysis.visualizegraph}}

\begin{fulllineitems}
\phantomsection\label{\detokenize{index:hicanalysis.visualizegraph.weights_distribuion}}
\pysigstartsignatures
\pysiglinewithargsret{\sphinxcode{\sphinxupquote{hicanalysis.visualizegraph.}}\sphinxbfcode{\sphinxupquote{weights\_distribuion}}}{\emph{\DUrole{n}{matrix}\DUrole{p}{:}\DUrole{w}{  }\DUrole{n}{ndarray}}, \emph{\DUrole{n}{savepath}\DUrole{p}{:}\DUrole{w}{  }\DUrole{n}{str\DUrole{w}{  }\DUrole{p}{|}\DUrole{w}{  }None}\DUrole{w}{  }\DUrole{o}{=}\DUrole{w}{  }\DUrole{default_value}{None}}}{}
\pysigstopsignatures
\sphinxAtStartPar
This function, given an adjacency matrix of a weighted network, plots and shows 
the distribution of the weights of the network. If the parameter \sphinxtitleref{savepath} is given 
the image will be saved in the location provided.


\subsection{Parameters}
\label{\detokenize{index:id13}}\begin{quote}
\begin{description}
\sphinxlineitem{matrix}{[}np.ndarray {]}
\sphinxAtStartPar
Adjacency matrix under the form of np.ndarray.

\sphinxlineitem{savepath}{[}Optional{[}str, path\sphinxhyphen{}like{]}, optional {]}
\sphinxAtStartPar
Parameter to chose the path and name where to save the file. If this
parameter is not given, the image will not be saved. Defaults to None.

\end{description}
\end{quote}

\end{fulllineitems}



\section{The degree\_distribution function}
\label{\detokenize{index:the-degree-distribution-function}}\index{degree\_distribution() (in module hicanalysis.visualizegraph)@\spxentry{degree\_distribution()}\spxextra{in module hicanalysis.visualizegraph}}

\begin{fulllineitems}
\phantomsection\label{\detokenize{index:hicanalysis.visualizegraph.degree_distribution}}
\pysigstartsignatures
\pysiglinewithargsret{\sphinxcode{\sphinxupquote{hicanalysis.visualizegraph.}}\sphinxbfcode{\sphinxupquote{degree\_distribution}}}{\emph{\DUrole{n}{network\_graph}\DUrole{p}{:}\DUrole{w}{  }\DUrole{n}{Graph}}, \emph{\DUrole{n}{savepath}\DUrole{p}{:}\DUrole{w}{  }\DUrole{n}{str\DUrole{w}{  }\DUrole{p}{|}\DUrole{w}{  }None}\DUrole{w}{  }\DUrole{o}{=}\DUrole{w}{  }\DUrole{default_value}{None}}}{}
\pysigstopsignatures
\sphinxAtStartPar
This function, given a network graph, plots and shows the distribution of the degree of the nodes of the network.
If the parameter \sphinxtitleref{savepath} is given the image will be saved in the location provided.


\subsection{Parameters}
\label{\detokenize{index:id14}}\begin{quote}
\begin{description}
\sphinxlineitem{network\_graph}{[}nx.Graph{]}
\sphinxAtStartPar
Network graph under the form of object of the Graph class of Networkx library.

\sphinxlineitem{savepath}{[}Optional{[}str, path\sphinxhyphen{}like{]}, optional {]}
\sphinxAtStartPar
Parameter to chose the path and name where to save the file. If this
parameter is not given, the image will not be saved. Defaults to None.

\end{description}
\end{quote}

\end{fulllineitems}



\section{The plot\_chromosome\_hics function}
\label{\detokenize{index:the-plot-chromosome-hics-function}}\index{plot\_chromosome\_hics() (in module hicanalysis.visualizegraph)@\spxentry{plot\_chromosome\_hics()}\spxextra{in module hicanalysis.visualizegraph}}

\begin{fulllineitems}
\phantomsection\label{\detokenize{index:hicanalysis.visualizegraph.plot_chromosome_hics}}
\pysigstartsignatures
\pysiglinewithargsret{\sphinxcode{\sphinxupquote{hicanalysis.visualizegraph.}}\sphinxbfcode{\sphinxupquote{plot\_chromosome\_hics}}}{\emph{\DUrole{n}{matrices}\DUrole{p}{:}\DUrole{w}{  }\DUrole{n}{ndarray}}, \emph{\DUrole{n}{names}\DUrole{p}{:}\DUrole{w}{  }\DUrole{n}{ndarray}}, \emph{\DUrole{n}{savepath}\DUrole{p}{:}\DUrole{w}{  }\DUrole{n}{str\DUrole{w}{  }\DUrole{p}{|}\DUrole{w}{  }None}\DUrole{w}{  }\DUrole{o}{=}\DUrole{w}{  }\DUrole{default_value}{None}}}{}
\pysigstopsignatures
\sphinxAtStartPar
This function produce an image containing the HiC matrices of the 22+2 chromosomes of the DNA.
The function don’t apply any logarithm to the data.
If the parameter \sphinxtitleref{savepath} is given the image will be saved in the location provided.


\subsection{Parameters}
\label{\detokenize{index:id15}}\begin{quote}
\begin{description}
\sphinxlineitem{matrices}{[}np.ndarray {]}
\sphinxAtStartPar
Array containing all the 24 matrices corresponding to the chromosomes

\sphinxlineitem{names}{[}np.ndarray{]}
\sphinxAtStartPar
Array containing the names to label each plot

\sphinxlineitem{savepath}{[}Optional{[}str, path\sphinxhyphen{}like{]}, optional {]}
\sphinxAtStartPar
Parameter to chose the path and name where to save the file. If this
parameter is not given, the image will not be saved. Defaults to None.

\end{description}
\end{quote}

\end{fulllineitems}



\section{The plot\_chromosomes\_histograms function}
\label{\detokenize{index:the-plot-chromosomes-histograms-function}}\index{plot\_chromosomes\_histograms() (in module hicanalysis.visualizegraph)@\spxentry{plot\_chromosomes\_histograms()}\spxextra{in module hicanalysis.visualizegraph}}

\begin{fulllineitems}
\phantomsection\label{\detokenize{index:hicanalysis.visualizegraph.plot_chromosomes_histograms}}
\pysigstartsignatures
\pysiglinewithargsret{\sphinxcode{\sphinxupquote{hicanalysis.visualizegraph.}}\sphinxbfcode{\sphinxupquote{plot\_chromosomes\_histograms}}}{\emph{\DUrole{n}{matrices}\DUrole{p}{:}\DUrole{w}{  }\DUrole{n}{ndarray}}, \emph{\DUrole{n}{names}\DUrole{p}{:}\DUrole{w}{  }\DUrole{n}{ndarray}}, \emph{\DUrole{n}{savepath}\DUrole{p}{:}\DUrole{w}{  }\DUrole{n}{str\DUrole{w}{  }\DUrole{p}{|}\DUrole{w}{  }None}\DUrole{w}{  }\DUrole{o}{=}\DUrole{w}{  }\DUrole{default_value}{None}}}{}
\pysigstopsignatures
\sphinxAtStartPar
This function produce an image containing the histograms of the absolute value 
of the eigenvalues of the 22+2 chromosomes of the DNA.
The function don’t apply any logarithm to the data.
If the parameter \sphinxtitleref{savepath} is given the image will be saved in the location provided.


\subsection{Parameters}
\label{\detokenize{index:id16}}\begin{quote}
\begin{description}
\sphinxlineitem{matrices}{[}np.ndarray {]}
\sphinxAtStartPar
Array containing all the 24 matrices corresponding to the chromosomes.

\sphinxlineitem{names}{[}np.ndarray {]}
\sphinxAtStartPar
Array containing the names to label each plot.

\sphinxlineitem{savepath}{[}Optional{[}str, path\sphinxhyphen{}like{]}, optional {]}
\sphinxAtStartPar
Parameter to chose the path and name where to save the file. If this
parameter is not given, the image will not be saved. Defaults to None.

\end{description}
\end{quote}

\end{fulllineitems}



\section{The plot\_matrix function}
\label{\detokenize{index:the-plot-matrix-function}}\index{plot\_matrix() (in module hicanalysis.visualizegraph)@\spxentry{plot\_matrix()}\spxextra{in module hicanalysis.visualizegraph}}

\begin{fulllineitems}
\phantomsection\label{\detokenize{index:hicanalysis.visualizegraph.plot_matrix}}
\pysigstartsignatures
\pysiglinewithargsret{\sphinxcode{\sphinxupquote{hicanalysis.visualizegraph.}}\sphinxbfcode{\sphinxupquote{plot\_matrix}}}{\emph{\DUrole{n}{matrix}\DUrole{p}{:}\DUrole{w}{  }\DUrole{n}{ndarray}}, \emph{\DUrole{n}{title}\DUrole{p}{:}\DUrole{w}{  }\DUrole{n}{str\DUrole{w}{  }\DUrole{p}{|}\DUrole{w}{  }None}\DUrole{w}{  }\DUrole{o}{=}\DUrole{w}{  }\DUrole{default_value}{\textquotesingle{}\textquotesingle{}}}, \emph{\DUrole{n}{savepath}\DUrole{p}{:}\DUrole{w}{  }\DUrole{n}{str\DUrole{w}{  }\DUrole{p}{|}\DUrole{w}{  }None}\DUrole{w}{  }\DUrole{o}{=}\DUrole{w}{  }\DUrole{default_value}{None}}}{}
\pysigstopsignatures
\sphinxAtStartPar
This function shows the image of the matrix with a colorbar. 
It is possible to set a title with the parameter title.
If the parameter \sphinxtitleref{savepath} is given the image will be saved in the location provided.


\subsection{Parameters}
\label{\detokenize{index:id17}}\begin{quote}
\begin{description}
\sphinxlineitem{matrix}{[}np.ndarray {]}
\sphinxAtStartPar
Matrix in the form of a np.ndarray.

\sphinxlineitem{title}{[}Optional{[}str{]}, optional  {]}
\sphinxAtStartPar
Title to put in the image. Defaults to ‘’.

\sphinxlineitem{savepath}{[}Optional{[}str, path\sphinxhyphen{}like{]}, optional {]}
\sphinxAtStartPar
Parameter to chose the path and name where to save the file. If this
parameter is not given, the image will not be saved. Defaults to None.

\end{description}
\end{quote}

\end{fulllineitems}



\section{The scatter\_plot function}
\label{\detokenize{index:the-scatter-plot-function}}\index{scatter\_plot() (in module hicanalysis.visualizegraph)@\spxentry{scatter\_plot()}\spxextra{in module hicanalysis.visualizegraph}}

\begin{fulllineitems}
\phantomsection\label{\detokenize{index:hicanalysis.visualizegraph.scatter_plot}}
\pysigstartsignatures
\pysiglinewithargsret{\sphinxcode{\sphinxupquote{hicanalysis.visualizegraph.}}\sphinxbfcode{\sphinxupquote{scatter\_plot}}}{\emph{\DUrole{n}{image1}\DUrole{p}{:}\DUrole{w}{  }\DUrole{n}{ndarray}}, \emph{\DUrole{n}{image2}\DUrole{p}{:}\DUrole{w}{  }\DUrole{n}{ndarray}}, \emph{\DUrole{n}{label1}\DUrole{p}{:}\DUrole{w}{  }\DUrole{n}{str\DUrole{w}{  }\DUrole{p}{|}\DUrole{w}{  }None}\DUrole{w}{  }\DUrole{o}{=}\DUrole{w}{  }\DUrole{default_value}{\textquotesingle{}Image 1\textquotesingle{}}}, \emph{\DUrole{n}{label2}\DUrole{p}{:}\DUrole{w}{  }\DUrole{n}{str\DUrole{w}{  }\DUrole{p}{|}\DUrole{w}{  }None}\DUrole{w}{  }\DUrole{o}{=}\DUrole{w}{  }\DUrole{default_value}{\textquotesingle{}Image 2\textquotesingle{}}}, \emph{\DUrole{n}{savepath}\DUrole{p}{:}\DUrole{w}{  }\DUrole{n}{str\DUrole{w}{  }\DUrole{p}{|}\DUrole{w}{  }None}\DUrole{w}{  }\DUrole{o}{=}\DUrole{w}{  }\DUrole{default_value}{None}}}{}
\pysigstopsignatures
\sphinxAtStartPar
This function generate a scatter plot to compare two images or matrices. Given 
two numpy.ndarray, and optionally the label to assign to each image, it plots 
the scatter plot of the values of the images. If the parameter \sphinxtitleref{savepath} is 
given the image will be saved in the location provided.


\subsection{Parameters}
\label{\detokenize{index:id18}}\begin{quote}
\begin{description}
\sphinxlineitem{image1}{[}np.ndarray {]}
\sphinxAtStartPar
First image, that lays on the x\sphinxhyphen{}axis.

\sphinxlineitem{image2}{[}np.ndarray  {]}
\sphinxAtStartPar
Second image, that lays on the y\sphinxhyphen{}axis.

\sphinxlineitem{label1 Optional{[}str{]}, optional  }
\sphinxAtStartPar
Label to assign to the first image. Defaults to ‘Image 1’.

\sphinxlineitem{label2 Optional{[}str{]}, optional  }
\sphinxAtStartPar
Label to assign to the second image. Defaults to ‘Image 2’.

\sphinxlineitem{savepath}{[}Optional{[}str, path\sphinxhyphen{}like{]}, optional  {]}
\sphinxAtStartPar
Parameter to chose the path and name where to save the file. If this
parameter is not given, the image will not be saved. Defaults to None.

\end{description}
\end{quote}

\end{fulllineitems}



\section{The histogram function}
\label{\detokenize{index:the-histogram-function}}\index{histogram() (in module hicanalysis.visualizegraph)@\spxentry{histogram()}\spxextra{in module hicanalysis.visualizegraph}}

\begin{fulllineitems}
\phantomsection\label{\detokenize{index:hicanalysis.visualizegraph.histogram}}
\pysigstartsignatures
\pysiglinewithargsret{\sphinxcode{\sphinxupquote{hicanalysis.visualizegraph.}}\sphinxbfcode{\sphinxupquote{histogram}}}{\emph{\DUrole{n}{data}}, \emph{\DUrole{n}{label}}, \emph{\DUrole{n}{xlabel}\DUrole{p}{:}\DUrole{w}{  }\DUrole{n}{str\DUrole{w}{  }\DUrole{p}{|}\DUrole{w}{  }None}\DUrole{w}{  }\DUrole{o}{=}\DUrole{w}{  }\DUrole{default_value}{\textquotesingle{}\textquotesingle{}}}, \emph{\DUrole{n}{ylabel}\DUrole{p}{:}\DUrole{w}{  }\DUrole{n}{str\DUrole{w}{  }\DUrole{p}{|}\DUrole{w}{  }None}\DUrole{w}{  }\DUrole{o}{=}\DUrole{w}{  }\DUrole{default_value}{\textquotesingle{}\textquotesingle{}}}, \emph{\DUrole{n}{xlim}\DUrole{p}{:}\DUrole{w}{  }\DUrole{n}{tuple\DUrole{w}{  }\DUrole{p}{|}\DUrole{w}{  }None}\DUrole{w}{  }\DUrole{o}{=}\DUrole{w}{  }\DUrole{default_value}{None}}, \emph{\DUrole{n}{title}\DUrole{p}{:}\DUrole{w}{  }\DUrole{n}{str}\DUrole{w}{  }\DUrole{o}{=}\DUrole{w}{  }\DUrole{default_value}{\textquotesingle{}\textquotesingle{}}}, \emph{\DUrole{n}{ylogscale}\DUrole{p}{:}\DUrole{w}{  }\DUrole{n}{bool}\DUrole{w}{  }\DUrole{o}{=}\DUrole{w}{  }\DUrole{default_value}{False}}, \emph{\DUrole{n}{savepath}\DUrole{p}{:}\DUrole{w}{  }\DUrole{n}{str\DUrole{w}{  }\DUrole{p}{|}\DUrole{w}{  }None}\DUrole{w}{  }\DUrole{o}{=}\DUrole{w}{  }\DUrole{default_value}{None}}}{}
\pysigstopsignatures
\sphinxAtStartPar
This function plot an histogram of the \sphinxtitleref{data} with a predefined \sphinxtitleref{darkgrid} style 
of seaborn. In the case \sphinxtitleref{data} is an array, it will be shown the histogram of 
the values, if \sphinxtitleref{data} is a sequence of array it will be shown an image 
containing all the corresponding histograms.With the \sphinxtitleref{data} parameter it is 
necessary to provide also a \sphinxtitleref{label} parameter, that must have the same 
dimension of \sphinxtitleref{data}. \sphinxtitleref{xlabel} and \sphinxtitleref{ylabel} are string that are used to set a 
label to the corresponding axis, while \sphinxtitleref{xlim} is a tuple that help to limits 
in the visualization of the plot. It is possible to set a title for the 
plot with the \sphinxtitleref{title} parameter and to set the log scale in y\sphinxhyphen{}axis with 
\sphinxtitleref{ylogscale} parameter.
If the parameter \sphinxtitleref{savepath} is given the image will be saved in the location provided.


\subsection{Parameters}
\label{\detokenize{index:id19}}\begin{quote}
\begin{description}
\sphinxlineitem{data}{[}(n,) array or sequence of (n,) arrays{]}
\sphinxAtStartPar
Input values, this takes either a single array or a sequence of arrays 
which are not required to be of the same length.

\sphinxlineitem{label}{[}str or sequence of str{]}
\sphinxAtStartPar
Names of input values, this takes either a single string or a sequence of 
strings depending on the dimension of the input data.

\sphinxlineitem{xlabel}{[}Optional{[}str{]}, optional{]}
\sphinxAtStartPar
Label to be shown in the x\sphinxhyphen{}axis of the plot. Default to ‘’.

\sphinxlineitem{ylabel}{[}Optional{[}str{]}, optional{]}
\sphinxAtStartPar
Label to be shown in the y\sphinxhyphen{}axis of the plot. Default to ‘’.

\sphinxlineitem{xlim}{[}Optional{[}tuple{]}, optional{]}
\sphinxAtStartPar
Tuple containing the lower and upper limits to display in the plot. Default 
to None

\sphinxlineitem{title}{[}Optional{[}str{]}, optional {]}
\sphinxAtStartPar
Title of the plot to visualize in the image. Defaults to ‘’.

\sphinxlineitem{ylogscale}{[}Optional{[}bool{]}, optional {]}
\sphinxAtStartPar
Parameter to set the log scale on y\sphinxhyphen{}axis. Defaults to False.

\sphinxlineitem{savepath}{[}Optional{[}str, path\sphinxhyphen{}like{]}, optional {]}
\sphinxAtStartPar
Parameter to chose the path and name where to save the file. If this
parameter is not given, the image will not be saved. Defaults to None.

\end{description}
\end{quote}

\end{fulllineitems}



\section{The show\_10\_projectors function}
\label{\detokenize{index:the-show-10-projectors-function}}\index{show\_10\_projectors() (in module hicanalysis.visualizegraph)@\spxentry{show\_10\_projectors()}\spxextra{in module hicanalysis.visualizegraph}}

\begin{fulllineitems}
\phantomsection\label{\detokenize{index:hicanalysis.visualizegraph.show_10_projectors}}
\pysigstartsignatures
\pysiglinewithargsret{\sphinxcode{\sphinxupquote{hicanalysis.visualizegraph.}}\sphinxbfcode{\sphinxupquote{show\_10\_projectors}}}{\emph{\DUrole{n}{projectors}\DUrole{p}{:}\DUrole{w}{  }\DUrole{n}{list}}, \emph{\DUrole{n}{savepath}\DUrole{p}{:}\DUrole{w}{  }\DUrole{n}{str\DUrole{w}{  }\DUrole{p}{|}\DUrole{w}{  }None}\DUrole{w}{  }\DUrole{o}{=}\DUrole{w}{  }\DUrole{default_value}{None}}}{}
\pysigstopsignatures
\sphinxAtStartPar
This function shows in a (2 rows x 5 columns) grid the ten projectors given as input.
If the parameter \sphinxtitleref{savepath} is given the image will be saved in the location provided.


\subsection{Parameters}
\label{\detokenize{index:id20}}\begin{description}
\sphinxlineitem{projectors}{[}list{]}
\sphinxAtStartPar
List containing ten matrices of the projectors to be shown.

\sphinxlineitem{savepath}{[}Optional{[}str, path\sphinxhyphen{}like{]}, optional{]}
\sphinxAtStartPar
Parameter to chose the path and name where to save the file. If this
parameter is not given, the image will not be saved. Defaults to None.

\end{description}

\end{fulllineitems}



\section{The plot\_matrix\_comparison function}
\label{\detokenize{index:the-plot-matrix-comparison-function}}\index{plot\_matrix\_comparison() (in module hicanalysis.visualizegraph)@\spxentry{plot\_matrix\_comparison()}\spxextra{in module hicanalysis.visualizegraph}}

\begin{fulllineitems}
\phantomsection\label{\detokenize{index:hicanalysis.visualizegraph.plot_matrix_comparison}}
\pysigstartsignatures
\pysiglinewithargsret{\sphinxcode{\sphinxupquote{hicanalysis.visualizegraph.}}\sphinxbfcode{\sphinxupquote{plot\_matrix\_comparison}}}{\emph{\DUrole{n}{matrix1}\DUrole{p}{:}\DUrole{w}{  }\DUrole{n}{ndarray}}, \emph{\DUrole{n}{matrix2}\DUrole{p}{:}\DUrole{w}{  }\DUrole{n}{ndarray}}, \emph{\DUrole{n}{title1}\DUrole{p}{:}\DUrole{w}{  }\DUrole{n}{str\DUrole{w}{  }\DUrole{p}{|}\DUrole{w}{  }None}\DUrole{w}{  }\DUrole{o}{=}\DUrole{w}{  }\DUrole{default_value}{\textquotesingle{}\textquotesingle{}}}, \emph{\DUrole{n}{title2}\DUrole{p}{:}\DUrole{w}{  }\DUrole{n}{str\DUrole{w}{  }\DUrole{p}{|}\DUrole{w}{  }None}\DUrole{w}{  }\DUrole{o}{=}\DUrole{w}{  }\DUrole{default_value}{\textquotesingle{}\textquotesingle{}}}, \emph{\DUrole{n}{pixels\_to\_be\_masked}\DUrole{p}{:}\DUrole{w}{  }\DUrole{n}{list\DUrole{w}{  }\DUrole{p}{|}\DUrole{w}{  }None}\DUrole{w}{  }\DUrole{o}{=}\DUrole{w}{  }\DUrole{default_value}{None}}, \emph{\DUrole{n}{savepath}\DUrole{p}{:}\DUrole{w}{  }\DUrole{n}{str\DUrole{w}{  }\DUrole{p}{|}\DUrole{w}{  }None}\DUrole{w}{  }\DUrole{o}{=}\DUrole{w}{  }\DUrole{default_value}{None}}}{}
\pysigstopsignatures
\sphinxAtStartPar
This function plots two matrices in the same image in order to compare them visually.
With the parameter \sphinxtitleref{pixels\_to\_be\_masked} it is possible to set to zero some specific pixels
of the first image to make it a more readable.
If the parameter \sphinxtitleref{savepath} is given the image will be saved in the location provided.


\subsection{Parameters}
\label{\detokenize{index:id21}}\begin{description}
\sphinxlineitem{matrix1}{[}np.ndarray{]}
\sphinxAtStartPar
Matrix containing the data of the first image.

\sphinxlineitem{matrix2}{[}np.ndarray{]}
\sphinxAtStartPar
Matrix containing the data of the second image.

\sphinxlineitem{title1}{[}Optional{[}str{]}, optional{]}
\sphinxAtStartPar
String containing the title of the first image. Defaults to ‘’.

\sphinxlineitem{title2}{[}Optional{[}str{]}, optional{]}
\sphinxAtStartPar
String containing the title of the second image. Defaults to ‘’.

\sphinxlineitem{pixels\_to\_be\_masked}{[}Optional{[}list{]}, optional{]}
\sphinxAtStartPar
List of tuples that contain the coordinates of the pixel to be masked of the first 
image. Defaults to None. Ex. With the list \sphinxtitleref{{[}(1, 2), (3, 4){]}} are masked the pixel 
whose coordinates are: \sphinxtitleref{(1, 2)} and \sphinxtitleref{(3, 4)}.

\sphinxlineitem{savepath}{[}Optional{[}bool{]}, optional{]}
\sphinxAtStartPar
Parameter to chose the path and name where to save the file. If this
parameter is not given, the image will not be saved. Defaults to None.

\end{description}

\end{fulllineitems}



\chapter{Indices and tables}
\label{\detokenize{index:indices-and-tables}}\begin{itemize}
\item {} 
\sphinxAtStartPar
\DUrole{xref,std,std-ref}{genindex}

\item {} 
\sphinxAtStartPar
\DUrole{xref,std,std-ref}{search}

\end{itemize}



\renewcommand{\indexname}{Index}
\printindex
\end{document}